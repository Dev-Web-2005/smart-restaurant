\documentclass[12pt]{article}

\usepackage{hcmus-report-template}
\usepackage[utf8]{inputenc}
\usepackage[vietnamese]{babel}
\usepackage[T5]{fontenc}
\usepackage{geometry}
\usepackage{graphicx}
\usepackage{xcolor}
\usepackage{tcolorbox}
\usepackage{listings}
\usepackage{enumitem}
\usepackage{booktabs}
\usepackage{longtable}
\usepackage{tabularx}
\usepackage{multirow}
\usepackage{hyperref}
\usepackage{amsmath}
\usepackage{amssymb}
\usepackage{float}

% ===== GEOMETRY =====
\geometry{
    left=2.5cm,
    right=2.5cm,
    top=3cm,
    bottom=3cm
}

% ===== COLORS =====
\definecolor{primarycolor}{RGB}{59,130,246}
\definecolor{secondarycolor}{RGB}{139,92,246}
\definecolor{successcolor}{RGB}{34,197,94}
\definecolor{warningcolor}{RGB}{234,179,8}
\definecolor{codebackground}{RGB}{40,44,52}
\definecolor{codecomment}{RGB}{92,99,112}
\definecolor{codekeyword}{RGB}{198,120,221}
\definecolor{codestring}{RGB}{152,195,121}
\definecolor{codefunction}{RGB}{97,175,239}

% ===== LISTINGS CONFIGURATION =====
\lstdefinelanguage{TypeScript}{
  keywords={const, let, var, function, return, if, else, for, while, import, export, default, from, async, await, class, interface, implements, extends, public, private, protected, readonly, new, try, catch, throw, type, enum},
  keywordstyle=\color{codekeyword}\bfseries,
  ndkeywords={@Injectable, @Controller, @Get, @Post, @Patch, @Delete, @Body, @Param, @Query, @UseGuards, Column, Entity, PrimaryGeneratedColumn, CreateDateColumn, UpdateDateColumn, ManyToOne, JoinColumn, Index},
  ndkeywordstyle=\color{warningcolor}\bfseries,
  identifierstyle=\color{white},
  sensitive=true,
  comment=[l]{//},
  morecomment=[s]{/*}{*/},
  commentstyle=\color{codecomment}\ttfamily,
  stringstyle=\color{codestring}\ttfamily,
  morestring=[b]',
  morestring=[b]",
  morestring=[b]`
}

\lstdefinelanguage{JavaScript}{
  keywords={const, let, var, function, return, if, else, for, while, import, export, default, from, async, await, class, extends, useState, useCallback, useMemo, memo, useRef, useQuery, useMutation},
  keywordstyle=\color{codekeyword}\bfseries,
  ndkeywords={Array, Object, String, Number, Boolean, Function, Math, Date, JSON, Promise, Map, Set},
  ndkeywordstyle=\color{codefunction}\bfseries,
  identifierstyle=\color{white},
  sensitive=true,
  comment=[l]{//},
  morecomment=[s]{/*}{*/},
  commentstyle=\color{codecomment}\ttfamily,
  stringstyle=\color{codestring}\ttfamily,
  morestring=[b]',
  morestring=[b]",
  morestring=[b]`
}

\lstset{
  backgroundcolor=\color{codebackground},
  basicstyle=\ttfamily\small\color{white},
  breakatwhitespace=false,
  breaklines=true,
  captionpos=b,
  extendedchars=true,
  frame=single,
  keepspaces=true,
  numbers=left,
  numbersep=10pt,
  numberstyle=\tiny\color{codecomment},
  rulecolor=\color{primarycolor},
  showspaces=false,
  showstringspaces=false,
  showtabs=false,
  tabsize=2,
  literate={ả}{{\h a}}1 {ế}{{\'\^e}}1 {ồ}{{\`\^o}}1 {ộ}{{\d\^o}}1
           {ứ}{{\'\h u}}1 {ử}{{\h\h u}}1 {ữ}{{\~\h u}}1 {ự}{{\d\h u}}1
           {ă}{{\u a}}1 {ắ}{{\'{\u a}}}1 {ằ}{{\`{\u a}}}1 {ẳ}{{\h{\u a}}}1
           {ẵ}{{\~{\u a}}}1 {ặ}{{\d{\u a}}}1
           {đ}{{\dj}}1 {Đ}{{\DJ}}1
           {ơ}{{\h o}}1 {ớ}{{\'{\h o}}}1 {ờ}{{\`{\h o}}}1 {ở}{{\h{\h o}}}1
           {ỡ}{{\~{\h o}}}1 {ợ}{{\d{\h o}}}1
           {ê}{{\^e}}1 {ề}{{\`\^e}}1 {ể}{{\h\^e}}1 {ễ}{{\~\^e}}1 {ệ}{{\d\^e}}1
           {ô}{{\^o}}1 {ố}{{\'{\^o}}}1 {ồ}{{\`{\^o}}}1 {ổ}{{\h{\^o}}}1
           {ỗ}{{\~{\^o}}}1 {ộ}{{\d{\^o}}}1
           {ư}{{\h u}}1 {ứ}{{\'{\h u}}}1 {ừ}{{\`{\h u}}}1 {ử}{{\h{\h u}}}1
           {ữ}{{\~{\h u}}}1 {ự}{{\d{\h u}}}1
           {ý}{{\'y}}1 {ỳ}{{\`y}}1 {ỷ}{{\h y}}1 {ỹ}{{\~y}}1 {ỵ}{{\d y}}1
           {á}{{\'a}}1 {à}{{\`a}}1 {ả}{{\h a}}1 {ã}{{\~a}}1 {ạ}{{\d a}}1
           {é}{{\'e}}1 {è}{{\`e}}1 {ẻ}{{\h e}}1 {ẽ}{{\~e}}1 {ẹ}{{\d e}}1
           {í}{{\'i}}1 {ì}{{\`i}}1 {ỉ}{{\h i}}1 {ĩ}{{\~i}}1 {ị}{{\d i}}1
           {ó}{{\'o}}1 {ò}{{\`o}}1 {ỏ}{{\h o}}1 {õ}{{\~o}}1 {ọ}{{\d o}}1
           {ú}{{\'u}}1 {ù}{{\`u}}1 {ủ}{{\h u}}1 {ũ}{{\~u}}1 {ụ}{{\d u}}1
}

% Hyperlink setup
\hypersetup{
    colorlinks=true,
    linkcolor=primarycolor,
    urlcolor=secondarycolor,
    citecolor=primarycolor
}

% Header and Footer
\pagestyle{fancy}
\fancyhf{}
\fancyhead[L]{\textbf{Smart Restaurant}}
\fancyhead[R]{\textit{Mô-đun Quản lý Bàn}}
\fancyfoot[C]{\thepage}
\renewcommand{\headrulewidth}{0.5pt}
\renewcommand{\footrulewidth}{0.5pt}
\setlength{\headheight}{31.89355pt}

% Disable indentation and set line spacing
\setlength{\parindent}{0pt}
\renewcommand{\baselinestretch}{1.5}
\sloppy

% Define course information
\newcommand{\coursename}{PHÁT TRIỂN ỨNG DỤNG WEB}
\newcommand{\reportname}{\small{BÁO CÁO DỰ ÁN CUỐI KỲ}}
\newcommand{\reporttitle}{SMART RESTAURANT - HỆ THỐNG ĐẶT MÓN QR}
\newcommand{\studentname}{%
  \begin{tabular}{@{} l r @{}}
    Lê Thành Công & (23120222)\\
    Nguyễn Hưng Thịnh & (23120200)\\
    Võ Cao Tâm Chính & (23120194)
  \end{tabular}%
}
\newcommand{\teachername}{ThS. Nguyễn Huy Khánh}

\lhead{\reporttitle}
\rhead{Trường Đại học Khoa học Tự nhiên - ĐHQG HCM\\ \coursename}

% ============ DOCUMENT ============
\begin{document}

\pagenumbering{roman}
\input{content/title.tex}

\tableofcontents
\pagebreak

\pagenumbering{arabic}
\setcounter{page}{1}

%------------------------------------------------------------
% FINAL PROJECT REPORT - Smart Restaurant QR Ordering System
%------------------------------------------------------------

\section{Mô tả Hệ thống}

\subsection{Tổng quan Dự án}
\textbf{Smart Restaurant} là một Hệ thống Quản lý Nhà hàng Đa khách hàng (Multi-Tenant) toàn diện với khả năng Đặt món qua Menu QR. Hệ thống cho phép nhà hàng quản lý menu, đơn hàng và bàn trong khi cho phép khách hàng đặt món trực tiếp bằng cách quét mã QR được đặt tại mỗi bàn.

\textbf{Điểm nổi bật:}
\begin{itemize}
    \item Kiến trúc Multi-Tenant hỗ trợ nhiều nhà hàng trên cùng một hệ thống
    \item Hơn 10 microservices (9 dịch vụ chính + dịch vụ bên ngoài)
    \item Cập nhật thời gian thực qua WebSocket và RabbitMQ
    \item Tích hợp thanh toán, email, và file storage
\end{itemize}

\textbf{Điểm nổi bật:}
\begin{itemize}
    \item Kiến trúc Multi-Tenant hỗ trợ nhiều nhà hàng trên cùng một hệ thống
    \item Hơn 10 microservices (9 dịch vụ chính + dịch vụ bên ngoài)
    \item Cập nhật thời gian thực qua WebSocket và RabbitMQ
    \item Tích hợp thanh toán, email, và file storage
\end{itemize}

\subsection{Mục tiêu Dự án}
\begin{itemize}
    \item Phát triển nền tảng SaaS đa khách hàng có khả năng mở rộng cho quản lý nhà hàng
    \item Cho phép truy cập menu và đặt món dựa trên QR không tiếp xúc
    \item Cung cấp theo dõi đơn hàng thời gian thực cho khách hàng và nhân viên bếp
    \item Triển khai bảng điều khiển quản trị toàn diện cho quản lý nhà hàng
    \item Hỗ trợ kiểm soát truy cập dựa trên vai trò cho các loại người dùng khác nhau
\end{itemize}

\subsection{Phạm vi Dự án}
\textbf{Các Tính năng Trong phạm vi:}
\begin{itemize}
    \item Đăng ký và tiếp nhận Khách hàng (Nhà hàng)
    \item Tạo mã QR cho mỗi bàn với token đã ký
    \item Quản lý menu (danh mục, món ăn, modifiers)
    \item Khách hàng đặt món qua quét QR
    \item Xử lý đơn hàng với state machine
    \item Hệ thống Hiển thị Bếp (KDS)
    \item Quy trình chấp nhận đơn hàng của Waiter
    \item Thông báo WebSocket thời gian thực
    \item Xác thực và phân quyền người dùng
\end{itemize}

\textbf{Ngoài phạm vi:}
\begin{itemize}
    \item Chương trình khách hàng thân thiết và khuyến mãi
    \item Phân tích đa chi nhánh
    \item Chế độ kiosk offline
    \item Tích hợp giao hàng
\end{itemize}

%------------------------------------------------------------
\section{Thông tin Nhóm}

\begin{table}[H]
\centering
\begin{tabular}{|c|l|l|c|}
\hline
\textbf{MSSV} & \textbf{Họ và Tên} & \textbf{Vai trò} & \textbf{Đóng góp \%} \\
\hline
23120222 & Lê Thành Công & Trưởng nhóm, Full-stack Developer & 33.4\% \\
\hline
23120200 & Nguyễn Hưng Thịnh & Trưởng phát triển Backend & 33.3\% \\
\hline
23120194 & Võ Cao Tâm Chính & Trưởng phát triển Frontend & 33.3\% \\
\hline
\end{tabular}
\caption{Thành viên Nhóm và Vai trò}
\end{table}

\subsection{Mô tả Vai trò}
\begin{itemize}
    \item \textbf{Lê Thành Công:} Thiết kế kiến trúc, API Gateway, Dịch vụ Identity, Dịch vụ Profile, WebSocket/Notification, Docker, CI/CD, DevOps
    \item \textbf{Nguyễn Hưng Thịnh:} Dịch vụ Product/Table, Dịch vụ Order, Dịch vụ Kitchen, Review, RabbitMQ, Tạo hóa đơn
    \item \textbf{Võ Cao Tâm Chính:} Frontend khách hàng, Bảng điều khiển Admin, Giao diện Waiter, Backend Waiter Service, Thiết kế UI/UX
\end{itemize}

%------------------------------------------------------------
\section{Theo dõi Kế hoạch Dự án}

\subsection{Tiến độ Phát triển}

\begin{table}[H]
\centering
\begin{tabularx}{\textwidth}{|c|X|c|}
\hline
\textbf{Sprint} & \textbf{Kết quả} & \textbf{Trạng thái} \\
\hline
Sprint 1 & Thiết lập dự án, Thiết kế Kiến trúc, Dịch vụ Identity & Hoàn thành \\
\hline
Sprint 2 & Dịch vụ Profile, Dịch vụ Product (CRUD Menu) & Hoàn thành \\
\hline
Sprint 3 & Dịch vụ Table, Tạo Mã QR & Hoàn thành \\
\hline
Sprint 4 & Dịch vụ Order, Chức năng Giỏ hàng & Hoàn thành \\
\hline
Sprint 5 & Dịch vụ Waiter, Dịch vụ Kitchen & Hoàn thành \\
\hline
Sprint 6 & Tích hợp WebSocket, Cập nhật thời gian thực & Hoàn thành \\
\hline
Sprint 7 & Tích hợp Frontend, Kiểm thử, Sửa lỗi & Hoàn thành \\
\hline
Sprint 8 & Triển khai Docker, Tài liệu, Báo cáo & Hoàn thành \\
\hline
\end{tabularx}
\caption{Kế hoạch Sprint và Tiến độ}
\end{table}

\subsection{Các cột mốc}
\begin{enumerate}
    \item \textbf{M1 - Nền tảng:} Thiết lập các dịch vụ cốt lõi (Identity, Profile, API Gateway)
    \item \textbf{M2 - Hệ thống Menu:} Dịch vụ Product với danh mục, món ăn, modifiers
    \item \textbf{M3 - Đặt món QR:} Quản lý bàn và tạo mã QR
    \item \textbf{M4 - Luồng Đơn hàng:} Vòng đời đơn hàng hoàn chỉnh với state machine
    \item \textbf{M5 - Thời gian thực:} Tích hợp WebSocket cho cập nhật trực tiếp
    \item \textbf{M6 - Triển khai:} Containerization và triển khai Docker
\end{enumerate}

%------------------------------------------------------------
\section{Phân tích Chức năng}

\subsection{Kiến trúc Hệ thống}
Hệ thống tuân theo \textbf{Kiến trúc Microservices Multi-Tenant} với các thành phần sau:
Hệ thống tuân theo \textbf{Kiến trúc Microservices Multi-Tenant} với các thành phần sau:

\textbf{Dịch vụ Chính (9 dịch vụ):}
\textbf{Dịch vụ Chính (9 dịch vụ):}
\begin{itemize}
    \item \textbf{API Gateway (Cổng 8888):} Điểm vào duy nhất, điều hướng, xác thực JWT
    \item \textbf{Dịch vụ Identity (Cổng 8084):} Xác thực, quản lý người dùng, RBAC
    \item \textbf{Dịch vụ Profile (Cổng 8081):} Quản lý hồ sơ người dùng và tenant
    \item \textbf{Dịch vụ Product (Cổng 8082):} Danh mục menu, món ăn, modifiers
    \item \textbf{Dịch vụ Table (Cổng 8083):} CRUD bàn, tạo mã QR
    \item \textbf{Dịch vụ Order (Cổng 8087):} Vòng đời đơn hàng, state machine
    \item \textbf{Dịch vụ Kitchen (Cổng 8086):} Chức năng KDS
    \item \textbf{Dịch vụ Waiter (Cổng 8088):} Chấp nhận/từ chối đơn hàng
    \item \textbf{Dịch vụ Notification (Cổng 8085):} Cập nhật WebSocket thời gian thực
\end{itemize}

\textbf{Dịch vụ Bên ngoài:}
\begin{itemize}
<<<<<<< Updated upstream
    \item \textbf{File Service:} Lưu trữ hình ảnh món ăn
    \item \textbf{Mail Service :} Gửi email xác thực, đặt lại mật khẩu
=======
    \item \textbf{File Service:} Lưu trữ hình ảnh món ăn (Render)
    \item \textbf{Mail Service:} Gửi email xác thực, đặt lại mật khẩu
>>>>>>> Stashed changes
    \item \textbf{Payment Service:} Xử lý thanh toán QR
    \item \textbf{Redis:} Cache giỏ hàng và session
    \item \textbf{RabbitMQ:} Message broker cho giao tiếp giữa các dịch vụ
\end{itemize}

\subsection{Các Chức năng Cốt lõi}

\subsubsection{1. Xác thực và Phân quyền}
\begin{itemize}
    \item Xác thực dựa trên JWT với Passport.js
    \item Kiểm soát truy cập dựa trên vai trò (Admin, User, Staff, Chef, Customer)
    \item Tích hợp đăng nhập xã hội (Google OAuth)
    \item Đặt lại mật khẩu qua email
    \item Kích hoạt tài khoản qua xác minh email
\end{itemize}

\subsubsection{2. Quản lý Menu}
\begin{itemize}
    \item Thao tác CRUD cho danh mục menu
    \item Món ăn với hình ảnh, giá, mô tả
    \item Nhóm modifier (Kích cỡ, Thêm, Topping) với điều chỉnh giá
    \item Quản lý tình trạng món ăn (Còn hàng, Không còn, Hết hàng)
    \item Chức năng xuất bản/hủy xuất bản
\end{itemize}

\subsubsection{3. Quản lý Bàn và Mã QR}
\begin{itemize}
    \item Tạo, chỉnh sửa, vô hiệu hóa bàn với sức chứa
    \item Tạo mã QR duy nhất cho mỗi bàn với token đã ký (HMAC-SHA256)
    \item Tải mã QR xuống dạng PNG để in
    \item Tái tạo mã QR để vô hiệu hóa mã cũ
    \item Phiên bản token để thu hồi ngay lập tức
\end{itemize}

\subsubsection{4. Quy trình Đặt món của Khách hàng}
\begin{itemize}
    \item Quét mã QR để truy cập menu nhà hàng
    \item Duyệt menu theo danh mục
    \item Tìm kiếm và lọc món ăn
    \item Thêm món vào giỏ hàng với modifiers
    \item Đặt món với yêu cầu đặc biệt
    \item Theo dõi trạng thái đơn hàng thời gian thực
\end{itemize}

\subsubsection{5. Xử lý Đơn hàng}
\begin{itemize}
    \item State machine đơn hàng: Bản nháp -> Đã gửi -> Đã nhận -> Đang chuẩn bị -> Sẵn sàng -> Hoàn thành
    \item Chấp nhận/từ chối món trong đơn hàng bởi Waiter
    \item Hệ thống Hiển thị Bếp cho chuẩn bị đơn hàng
    \item Cập nhật trạng thái thời gian thực qua WebSocket
    \item Theo dõi lịch sử đơn hàng
\end{itemize}

\subsubsection{6. Tính năng Thời gian thực}
\begin{itemize}
    \item Tích hợp WebSocket (Socket.IO)
    \item Cập nhật trạng thái đơn hàng trực tiếp cho khách hàng
    \item Thông báo bếp cho đơn hàng mới
    \item Thông báo waiter cho đơn hàng chờ xử lý
    \item RabbitMQ cho giao tiếp giữa các dịch vụ
\end{itemize}

\subsection{Các Luồng Người dùng}

\textbf{Luồng Đặt món của Khách hàng:}
\begin{enumerate}
    \item Khách hàng quét mã QR tại bàn
    \item Xác thực token và tải menu
    \item Duyệt menu và thêm món vào giỏ hàng
    \item Gửi đơn hàng với ghi chú tùy chọn
    \item Waiter nhận thông báo
    \item Waiter chấp nhận/từ chối món
    \item Bếp nhận món đã chấp nhận
    \item Bếp cập nhật trạng thái chuẩn bị
    \item Khách hàng nhận cập nhật thời gian thực
    \item Đơn hàng được đánh dấu hoàn thành
\end{enumerate}

%------------------------------------------------------------
\section{Thiết kế Cơ sở dữ liệu}

\subsection{Kiến trúc Cơ sở dữ liệu}
Mỗi microservice có cơ sở dữ liệu PostgreSQL riêng để cô lập dữ liệu:
\begin{itemize}
    \item \texttt{identity\_db} - Người dùng, Vai trò, Quyền hạn
    \item \texttt{profile\_db} - Hồ sơ người dùng, Thông tin Tenant
    \item \texttt{product\_db} - Danh mục, Món ăn, Modifiers
    \item \texttt{table\_db} - Bàn, Token QR
    \item \texttt{order\_db} - Đơn hàng, Món trong đơn, Lịch sử trạng thái
\end{itemize}

\subsection{Biểu đồ ER}

\begin{figure}[H]
\centering
\begin{tcolorbox}[colback=gray!5, colframe=gray!50, width=\textwidth]
\begin{verbatim}
TENANT ||--o{ USER : has
TENANT ||--o{ TABLE : has
TENANT ||--o{ MENU_CATEGORY : has
TENANT ||--o{ MENU_ITEM : has
TENANT ||--o{ ORDER : has
ORDER ||--o{ ORDER_ITEM : contains
MENU_CATEGORY ||--o{ MENU_ITEM : groups
MENU_ITEM ||--o{ MODIFIER_OPTION : offers
TABLE ||--o{ ORDER : places
\end{verbatim}
\end{tcolorbox}
\caption{Mối quan hệ Thực thể}
\end{figure}

\subsection{Các Thực thể Chính}

\begin{table}[H]
\centering
\begin{tabularx}{\textwidth}{|l|X|}
\hline
\textbf{Thực thể} & \textbf{Các trường Chính} \\
\hline
TENANT & id, name, email, timezone, created\_at \\
\hline
USER & id, tenant\_id, email, role, created\_at \\
\hline
TABLE & id, tenant\_id, name, capacity, location, qr\_token\_version \\
\hline
MENU\_CATEGORY & id, tenant\_id, name, description, published \\
\hline
MENU\_ITEM & id, category\_id, name, price, image\_url, available \\
\hline
MODIFIER\_OPTION & id, item\_id, group\_name, label, price\_delta, type \\
\hline
ORDER & id, tenant\_id, table\_id, status, total, currency \\
\hline
ORDER\_ITEM & id, order\_id, item\_id, quantity, modifiers, unit\_price \\
\hline
\end{tabularx}
\caption{Tóm tắt Trường Thực thể}
\end{table}

\subsection{Cô lập Dữ liệu Đa khách hàng}
Hệ thống triển khai Row-Level Security (RLS) với \texttt{tenant\_id} trên tất cả các bảng thuộc sở hữu của tenant để đảm bảo cô lập dữ liệu hoàn toàn giữa các nhà hàng.

%------------------------------------------------------------
\section{Thiết kế UI/UX}

\subsection{Nguyên tắc Thiết kế}
\begin{itemize}
    \item \textbf{Ưu tiên Mobile:} Giao diện đặt món khách hàng tối ưu cho thiết bị di động
    \item \textbf{Tối ưu Desktop:} Bảng điều khiển quản trị thiết kế cho sử dụng desktop
    \item \textbf{Glassmorphism:} Phong cách UI hiện đại với hiệu ứng kính cho các trang QR
    \item \textbf{Responsive:} Thích ứng với nhiều kích thước màn hình
\end{itemize}

\subsection{Ngăn xếp Công nghệ}
\begin{itemize}
    \item \textbf{Frontend:} React 19, Vite, TailwindCSS 4
    \item \textbf{Thành phần UI:} Thành phần tùy chỉnh với biểu tượng Lucide
    \item \textbf{Hiệu ứng chuyển động:} Framer Motion
    \item \textbf{Biểu đồ:} Recharts cho phân tích, ReactFlow
    \item \textbf{Quản lý Trạng thái:} React Context API
\end{itemize}

\subsection{Các Bố cục Ứng dụng}

\subsubsection{1. Giao diện Khách hàng}
\begin{itemize}
    \item Duyệt menu với bộ lọc danh mục
    \item Modal chi tiết món với modifiers
    \item Thanh bên giỏ hàng
    \item Trang theo dõi trạng thái đơn hàng
\end{itemize}

\subsubsection{2. Bảng điều khiển Quản trị}
\begin{itemize}
    \item Điều hướng thanh bên
    \item Quản lý menu (thao tác CRUD)
    \item Quản lý bàn với tạo QR
    \item Danh sách và quản lý đơn hàng
    \item Quản lý tài khoản nhân viên
\end{itemize}

\subsubsection{3. Hệ thống Hiển thị Bếp (KDS)}
\begin{itemize}
    \item Thẻ đơn hàng với chi tiết món
    \item Nút cập nhật trạng thái
    \item Đơn hàng đến thời gian thực
    \item Đồng hồ đếm cho chuẩn bị đơn hàng
\end{itemize}

\subsubsection{4. Giao diện Waiter}
\begin{itemize}
    \item Danh sách đơn hàng chờ xử lý
    \item Thao tác Chấp nhận/Từ chối món
    \item Tổng quan trạng thái đơn hàng
\end{itemize}

\subsection{Ảnh chụp màn hình} [CHÍNH]
% [TASK] Insert UI screenshots here
\begin{tcolorbox}[colback=yellow!10, colframe=yellow!50!black, title=Yêu cầu Ảnh chụp màn hình UI]
[TASK] Chèn ảnh chụp màn hình của các giao diện sau:
\begin{enumerate}
    \item Trang duyệt menu khách hàng
    \item Giỏ hàng và thanh toán khách hàng
    \item Tổng quan bảng điều khiển quản trị
    \item Trang quản lý menu
    \item Quản lý bàn với mã QR
    \item Hệ thống Hiển thị Bếp
    \item Quản lý đơn hàng Waiter
    \item Trang Đăng nhập/Đăng ký
\end{enumerate}
\end{tcolorbox}

%------------------------------------------------------------
\section{Hướng dẫn}

\subsection{Yêu cầu Hệ thống}
\begin{itemize}
    \item \textbf{Node.js:} v18+ (khuyến nghị LTS)
    \item \textbf{PostgreSQL:} v14+
    \item \textbf{Redis:} v6+
    \item \textbf{Docker:} v20+ (tùy chọn, cho triển khai containerized)
    \item \textbf{Git:} Để sao chép repository
\end{itemize}

\subsection{Thiết lập Phát triển Cục bộ}

\subsubsection{Bước 1: Sao chép Repository}
\begin{verbatim}
git clone https://github.com/Dev-Web-2005/smart-restaurant.git
cd smart-restaurant
\end{verbatim}

\subsubsection{Bước 2: Thiết lập Biến Môi trường}
Tạo tệp \texttt{.env} cho mỗi dịch vụ dựa trên \texttt{.env.example}.

\textbf{Cấu hình cơ sở dữ liệu:}
\begin{verbatim}
DATABASE_HOST=localhost
DATABASE_PORT=5432
DATABASE_NAME=identity_db
DATABASE_USER=postgres
DATABASE_PASSWORD=postgres123
\end{verbatim}

\textbf{Cấu hình JWT:}
\begin{verbatim}
JWT_SECRET=your-secret-key
JWT_EXPIRES_IN=7d
\end{verbatim}

\textbf{Cấu hình Redis:}
\begin{verbatim}
REDIS_HOST=localhost
REDIS_PORT=6379
\end{verbatim}

\subsubsection{Bước 3: Cài đặt Phụ thuộc}
\textbf{Các dịch vụ Backend:}
\begin{verbatim}
cd src/backend/api-gateway && npm install
cd ../identity && npm install
cd ../product && npm install
cd ../table && npm install
cd ../order && npm install
cd ../kitchen && npm install
cd ../waiter && npm install
cd ../notification && npm install
\end{verbatim}

\textbf{Frontend:}
\begin{verbatim}
cd src/frontend && npm install
\end{verbatim}

\subsubsection{Bước 4: Khởi động Dịch vụ}
\textbf{Sử dụng tệp batch được cung cấp:}
\begin{verbatim}
./run.bat
\end{verbatim}

\textbf{Hoặc khởi động thủ công từng dịch vụ:}
\begin{verbatim}
cd src/backend/api-gateway && npm run start:dev
cd src/backend/identity && npm run start:dev
cd src/backend/profile && npm run start:dev
cd src/backend/product && npm run start:dev
cd src/backend/table && npm run start:dev
cd src/backend/order && npm run start:dev
cd src/backend/kitchen && npm run start:dev
cd src/backend/waiter && npm run start:dev
cd src/backend/notification && npm run start:dev
cd src/frontend && npm run dev
\end{verbatim}

\subsection{Triển khai Docker}

\subsubsection{Khởi động Nhanh với Docker Compose}
\textbf{Xây dựng và khởi động tất cả dịch vụ:}
\begin{verbatim}
docker-compose up -d --build
\end{verbatim}

\textbf{Xem logs:}
\begin{verbatim}
docker-compose logs -f
\end{verbatim}

\textbf{Dừng tất cả dịch vụ:}
\begin{verbatim}
docker-compose down
\end{verbatim}

\subsubsection{Cổng Dịch vụ}
\begin{table}[H]
\centering
\begin{tabular}{|l|c|}
\hline
\textbf{Dịch vụ} & \textbf{Cổng} \\
\hline
Frontend & 5173 \\
\hline
API Gateway & 8888 \\
\hline
Dịch vụ Profile & 8081 \\
\hline
Dịch vụ Product & 8082 \\
\hline
Dịch vụ Table & 8083 \\
\hline
Dịch vụ Identity & 8084 \\
\hline
Dịch vụ Notification & 8085 \\
\hline
Dịch vụ Kitchen & 8086 \\
\hline
Dịch vụ Order & 8087 \\
\hline
Dịch vụ Waiter & 8088 \\
\hline
\end{tabular}
\caption{Ánh xạ Cổng Dịch vụ}
\end{table}

\subsection{Triển khai Production}

% [TASK] Add production URL if available
\begin{tcolorbox}[colback=yellow!10, colframe=yellow!50!black, title=URL Production]
[TASK] Chèn URL triển khai production tại đây (nếu được triển khai lên hosting công cộng).
\end{tcolorbox}

\subsubsection{Các bước Triển khai}
\begin{enumerate}
    \item Đặt biến môi trường production
    \item Xây dựng Docker images cho production
    \item Đẩy images lên container registry
    \item Triển khai sử dụng Docker Compose hoặc Kubernetes
    \item Cấu hình reverse proxy (Nginx) cho HTTPS
    \item Thiết lập sao lưu cơ sở dữ liệu
\end{enumerate}

\subsection{Hướng dẫn Sử dụng Hệ thống}

[CHÍNH] Thầy vào đọc chỉnh lại giúp em

\subsubsection{Quy trình Onboarding - Chủ Nhà hàng}

\textbf{Bước 1: Đăng ký Tài khoản}
\begin{enumerate}
    \item Truy cập \url{http://localhost:3000/register} (hoặc production URL)
    \item Điền thông tin: Tên nhà hàng, Email, Mật khẩu (tối thiểu 8 ký tự, bao gồm chữ hoa, số, ký tự đặc biệt)
    \item Chọn múi giờ và loại tiền tệ (VND, USD, EUR)
    \item Xác nhận email qua link được gửi đến hộp thư
    \item Hệ thống tự động tạo tenant và gán role \texttt{ADMIN}
\end{enumerate}

\textbf{Bước 2: Cấu hình Menu}
\begin{enumerate}
    \item Đăng nhập vào Admin Dashboard: \url{http://localhost:3000/admin}
    \item Tạo Categories: Navigate đến Menu Management → Categories → Add Category
    \begin{itemize}
        \item Nhập tên danh mục (Appetizers, Main Courses, Desserts, Beverages)
        \item Upload icon cho danh mục
        \item Set thứ tự hiển thị (display order)
    \end{itemize}
    \item Thêm Menu Items: Menu Management → Items → Add Item
    \begin{itemize}
        \item Nhập tên món, giá, mô tả
        \item Upload hình ảnh (tối đa 5 ảnh, mỗi ảnh < 5MB)
        \item Chọn danh mục
        \item Thời gian chuẩn bị (preparation time) tính bằng phút
        \item Trạng thái: Available / Unavailable / Sold Out
    \end{itemize}
    \item Tạo Modifiers: Trong Item Details → Modifier Groups
    \begin{itemize}
        \item Tạo groups như "Size" (Small, Medium, Large)
        \item "Extras" (Extra Cheese +\$2, Bacon +\$3)
        \item "Toppings" với price delta (có thể âm cho discounts)
        \item Set selection rules: required, multiple allowed, min/max selections
    \end{itemize}
\end{enumerate}

\textbf{Bước 3: Quản lý Bàn và QR Codes}
\begin{enumerate}
    \item Navigate: Table Management → Add Table
    \item Nhập thông tin:
    \begin{itemize}
        \item Table Name/Number (T01, T02, VIP-01)
        \item Capacity (số ghế)
        \item Location/Area (Main Hall, Terrace, Private Room)
        \item Status: Active / Inactive
    \end{itemize}
    \item Generate QR Code: Click "Generate QR" button
    \begin{itemize}
        \item Hệ thống tạo signed JWT token với payload: \texttt{\{tenant\_id, table\_id, version\}}
        \item Token được ký bằng HMAC-SHA256 với secret key
        \item QR code encode URL: \texttt{https://your-domain.com/order?token=<JWT>}
    \end{itemize}
    \item Download QR: Chọn định dạng PNG hoặc PDF
    \item In và đặt QR code tại bàn tương ứng
    \item Để vô hiệu hóa QR cũ: Click "Regenerate QR" → Version increment → Old tokens invalid
\end{enumerate}

\textbf{Bước 4: Quản lý Nhân viên}
\begin{enumerate}
    \item Navigate: Staff Management → Add Staff
    \item Chọn Role:
    \begin{itemize}
        \item \texttt{WAITER}: Quyền quản lý orders, accept/reject items
        \item \texttt{KITCHEN}: Truy cập KDS, update item preparation status
        \item \texttt{ADMIN}: Full access to management dashboard
    \end{itemize}
    \item Nhập thông tin: Full Name, Email, Temporary Password
    \item Gửi email activation cho nhân viên
    \item Nhân viên login lần đầu và đổi mật khẩu
\end{enumerate}

\subsubsection{Quy trình Đặt món - Khách hàng}

\textbf{Luồng Ordering (Dine-in):}
\begin{enumerate}
    \item \textbf{Scan QR Code:} Sử dụng camera smartphone quét mã QR tại bàn
    \item \textbf{Token Validation:} 
    \begin{itemize}
        \item Frontend gửi token đến API Gateway
        \item Table Service verify signature và expiration
        \item Kiểm tra table active status và token version
        \item Return: tenant\_id, table\_id, restaurant info
    \end{itemize}
    \item \textbf{Browse Menu:}
    \begin{itemize}
        \item Hiển thị menu của tenant tương ứng
        \item Filter theo categories
        \item Search by item name
        \item Sort by popularity, price, hoặc chef recommendation
    \end{itemize}
    \item \textbf{Item Selection:}
    \begin{itemize}
        \item Click vào item để xem chi tiết
        \item Chọn modifiers (Size, Extras, Special requests)
        \item Nhập số lượng
        \item Add to Cart
    \end{itemize}
    \item \textbf{Cart Review:}
    \begin{itemize}
        \item View tất cả items trong cart
        \item Update quantity hoặc remove items
        \item Auto-calculate: subtotal, modifiers cost, total
        \item Cart được persist trong localStorage và backend session
    \end{itemize}
    \item \textbf{Place Order:}
    \begin{itemize}
        \item Optional: Nhập guest name
        \item Thêm special instructions (No onions, Extra spicy, Allergy notes)
        \item Submit order
        \item Order status: DRAFT → SUBMITTED
    \end{itemize}
    \item \textbf{Real-time Tracking:}
    \begin{itemize}
        \item WebSocket connection established
        \item Subscribe to order updates
        \item Nhận notifications:
        \begin{itemize}
            \item Waiter accepted order → RECEIVED
            \item Kitchen started preparing → PREPARING
            \item Item ready → READY
            \item Order completed → COMPLETED
        \end{itemize}
        \item Hiển thị progress bar và estimated time
    \end{itemize}
    \item \textbf{Add More Items:}
    \begin{itemize}
        \item Khách hàng có thể tiếp tục order trong cùng session
        \item New items được thêm vào existing order (không tạo order mới)
        \item Luồng approval tương tự
    \end{itemize}
\end{enumerate}

\subsubsection{Quy trình Xử lý Đơn - Waiter}

\begin{enumerate}
    \item \textbf{Login:} Truy cập \url{http://localhost:3000/waiter}
    \item \textbf{Dashboard Overview:}
    \begin{itemize}
        \item Pending Orders: Đơn hàng mới chờ review
        \item Active Tables: Bàn đang có khách
        \item Notifications: Real-time alerts qua WebSocket
    \end{itemize}
    \item \textbf{Review Order:}
    \begin{itemize}
        \item Click vào pending order
        \item Xem chi tiết: table number, items, modifiers, special requests
        \item Verify item availability với kitchen
    \end{itemize}
    \item \textbf{Accept/Reject Items:}
    \begin{itemize}
        \item Accept: Item forward đến Kitchen
        \item Reject: Chọn lý do (Out of stock, Kitchen busy, Invalid request)
        \item Partial accept: Accept một số items, reject một số khác
        \item Customer nhận notification về rejected items qua WebSocket
    \end{itemize}
    \item \textbf{Monitor Progress:}
    \begin{itemize}
        \item Theo dõi kitchen preparation status
        \item Nhận alert khi items READY
        \item Serve items đến table
        \item Update status: SERVED
    \end{itemize}
\end{enumerate}

\subsubsection{Quy trình Chuẩn bị - Kitchen Staff}

\begin{enumerate}
    \item \textbf{Access KDS:} \url{http://localhost:3000/kitchen}
    \item \textbf{Kitchen Display System Interface:}
    \begin{itemize}
        \item Order cards hiển thị theo thứ tự thời gian
        \item Mỗi card chứa:
        \begin{itemize}
            \item Order number và table number
            \item Item list với modifiers
            \item Special instructions (highlighted)
            \item Preparation timer (đếm ngược từ item prep time)
        \end{itemize}
    \end{itemize}
    \item \textbf{Start Preparation:}
    \begin{itemize}
        \item Click "Start" button trên item card
        \item Status: QUEUED → PREPARING
        \item Timer bắt đầu đếm ngược
        \item Notification gửi đến customer qua WebSocket
    \end{itemize}
    \item \textbf{Mark as Ready:}
    \begin{itemize}
        \item Click "Ready" button khi hoàn thành
        \item Status: PREPARING → READY
        \item Sound alert cho waiter
        \item Card di chuyển sang "Ready for Pickup" column
    \end{itemize}
    \item \textbf{Alerts và Priority:}
    \begin{itemize}
        \item Items vượt quá prep time: Highlighted đỏ
        \item VIP orders: Gold border
        \item Urgent orders: Flashing animation
    \end{itemize}
\end{enumerate}

\subsection{Xử lý Sự cố Thường gặp}

\subsubsection{Lỗi Kết nối Database}

\textbf{Triệu chứng:} Service không khởi động, log hiển thị "ECONNREFUSED" hoặc "Connection timeout"

\textbf{Nguyên nhân:}
\begin{itemize}
    \item PostgreSQL service chưa được khởi động
    \item Sai thông tin credentials trong .env
    \item Firewall block port 5432
\end{itemize}

\textbf{Giải pháp:}
\begin{lstlisting}[language=bash]
# Kiểm tra PostgreSQL running
sudo systemctl status postgresql  # Linux
# Hoặc Task Manager → Services → postgresql-x64-14

# Test connection
psql -h localhost -U postgres -d identity_db

# Verify credentials trong .env
cat src/backend/identity/.env | grep DATABASE

# Cho phép port qua firewall (Windows)
netsh advfirewall firewall add rule name="PostgreSQL" dir=in action=allow protocol=TCP localport=5432
\end{lstlisting}

\subsubsection{Port Already in Use}

\textbf{Triệu chứng:} "Error: listen EADDRINUSE: address already in use :::3001"

\textbf{Giải pháp:}
\begin{lstlisting}[language=bash]
# Windows: Tìm process đang sử dụng port
netstat -ano | findstr :3001
taskkill /PID <process_id> /F

# Linux/Mac
lsof -i :3001
kill -9 <PID>

# Hoặc thay đổi port trong .env
PORT=3011
\end{lstlisting}

\subsubsection{JWT Token Invalid}

\textbf{Triệu chứng:} API trả về "Unauthorized" hoặc "Invalid token"

\textbf{Giải pháp:}
\begin{itemize}
    \item Verify JWT\_SECRET giống nhau giữa Identity Service và API Gateway
    \item Kiểm tra token expiration: JWT\_EXPIRES\_IN
    \item Clear browser cookies và localStorage
    \item Re-login để lấy token mới
\end{itemize}

\subsubsection{WebSocket Connection Failed}

\textbf{Triệu chứng:} Real-time updates không hoạt động

\textbf{Giải pháp:}
\begin{lstlisting}[language=bash]
# Verify Notification Service running
curl http://localhost:3007/health

# Check WebSocket endpoint
# Frontend console: "WebSocket connection to 'ws://localhost:3007' failed"

# Verify CORS configuration trong Notification Service
# main.ts:
app.enableCors({
  origin: ['http://localhost:3000', 'https://your-production-domain.com'],
  credentials: true
});
\end{lstlisting}

\subsubsection{Docker Containers Exit Immediately}

\textbf{Giải pháp:}
\begin{lstlisting}[language=bash]
# Xem logs để identify error
docker-compose logs service-name

# Common issues:
# 1. Missing environment variables
docker-compose config  # Verify env interpolation

# 2. Database not ready
# Ensure healthcheck và depends_on configured correctly

# 3. Application crash
docker exec -it service-name npm run start:dev  # Run interactive để debug
\end{lstlisting}
%------------------------------------------------------------
\section{Kết luận}

Hệ thống Đặt món QR cho Nhà hàng Thông minh đã triển khai thành công một giải pháp toàn diện cho hoạt động nhà hàng hiện đại. Các thành tựu chính bao gồm:

\begin{itemize}
    \item Kiến trúc microservices Multi-Tenant với hơn 10 dịch vụ (9 dịch vụ chính + dịch vụ bên ngoài)
    \item Hỗ trợ nhiều nhà hàng trên cùng một hệ thống với cô lập dữ liệu
    \item Đặt món dựa trên QR bảo mật với token đã ký HMAC-SHA256
    \item Cập nhật thời gian thực qua WebSocket và RabbitMQ
    \item Kiến trúc microservices Multi-Tenant với hơn 10 dịch vụ (9 dịch vụ chính + dịch vụ bên ngoài)
    \item Hỗ trợ nhiều nhà hàng trên cùng một hệ thống với cô lập dữ liệu
    \item Đặt món dựa trên QR bảo mật với token đã ký HMAC-SHA256
    \item Cập nhật thời gian thực qua WebSocket và RabbitMQ
    \item Kiểm soát truy cập dựa trên vai trò cho tất cả loại người dùng
    \item Tích hợp dịch vụ bên ngoài: File Service, Mail Service, Payment Service
    \item Tích hợp dịch vụ bên ngoài: File Service, Mail Service, Payment Service
    \item Containerization với Docker để triển khai dễ dàng
    \item Thiết kế responsive ưu tiên mobile
\end{itemize}

Hệ thống giải quyết các thách thức thực tế của nhà hàng bằng cách giảm thiểu lỗi đặt món, cải thiện hiệu suất bếp và nâng cao trải nghiệm khách hàng thông qua đặt món không tiếp xúc.

\end{document}